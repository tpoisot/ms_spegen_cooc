Each bipartite network is represented as its adjacency matrix $\mathbf{M}$
with $T$ rows (for the upper level, *i.e.* ectoparasites, herbivores,
and pollinators) and $L$ columns (for the lower level, *i.e.* animal hosts
and plants being eaten or pollinated). In each network, $\mathbf{M}_{ij}$
represents the existence of an interaction between species $i$ and the
species $j$ [@dunne_network_2006]. For each network, we calculate size
($Z=L\times T$), and connectance ($\mathrm{Co}$, the proportion of established
interactions). We focus our analyses on the upper trophic level, since we
have more knowledge of specialization mechanisms for these organisms [*e.g.*
@futuyma_evolution_1988]. Nestedness is calculated using the NODF (Nestedness
based on Overlap and Decreasing Fill) measure [@almeida-neto_consistent_2008]
, which is insensitive to network shape (asymmetry in the number of species
at each of the two trophic levels) and size. Modularity was estimated using
the LP-BRIM method [@liu_community_2010], which both increases modularity
detection compared to the adaptive BRIM method, and is less computationally
intensive [@barber_detecting_2009]. For each network, we retained the highest
modularity [$Q_{bip}$ *sensu* @barber_modularity_2007] observed in a total
of 1000 replicate runs.
