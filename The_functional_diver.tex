The functional diversity of ecological communities emerges through
the simultaneous occurrence of species with contrasted resource use
[@poisot2011a], habitat selection [@devictor2010], and interactions
[@bascompte2003;@weiner2012]. Both empirical and theoretical studies have
shown how different degrees of niche partitioning can promote functional
diversity [@diaz2001; @petchey2002; @ackerly2007]. However, the co-occurrence
of specialist and generalist species has received considerably less
attention. The majority of studies seeking to understand the conditions
for co-occurrence between populations of specialists and generalists in
both biotic (*e.g.* predator–prey, host–parasite) and abiotic (*e.g.*
habitat choice) interactions have focused on small communities [@wilson1994;
@hochberg1990a; @demeeus1995; @egas2004; @abrams2006; @ravigne2009]. Approaches
based on model analysis or controlled experiments have two features impeding
their generalization to large communities. First, the number of interacting
organisms is often kept low, either to facilitate model analysis or because of
experimental constraints. In practice, this means that studies investigating
the co-occurrence of species with contrasted specificities assume pure
specialism *vs.* pure specialism, whereas natural systems exhibit more of
a continuum [@forister_revisiting_2012; @poisot_structure_2013]. Second,
it is unclear to what extent results can be scaled up to more realistic
communities. @stouffer_understanding_2009 showed that because adding species
and interactions increases the potential for complex population dynamical
feedbacks, complete, realistic networks tend to exhibit different behaviors
than simple modules (*i.e.* those typically used in models or experiments),
begging for an analysis of co-occurrence in empirical communities.
