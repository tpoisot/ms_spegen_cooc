Network theory offers powerful tools to characterize the complexity of
ecological communities [@proulx_network_2005]. In a species interaction
network, each species is a node, and each interaction is an edge, connecting
a pair of nodes. From a network perspective, measuring the specificity
of a species can be performed by counting the number of links it has with
other species, or to study the distribution of the strengths of such links
[@poisot_comparative_2012]. Previous work described the degree-distribution
of empirical networks, *i.e.* the fact that there is a continuum of species,
ranging from highly specialized to generalists [@otto_allometric_2007]. While
much is known about the factors (*e.g.* biotic [@thrall_coevolution_2007],
abiotic [@ravigne_live_2009; @forister_revisiting_2012], developmental and
physiological [@ferry-graham_using_2002]) driving the specialization of single
species, less is known about the spectrum of specificities that can co-occur
in large ecological networks, and reasons for different spectra. As the
co-occurrence between specialized and generalized species is key to maintaining
functional diversity [@devictor_defining_2010] or promoting community stability
[@hassell_generalist_1986], there is a need to investigate the maintenance of
species with different specificities. In this paper, we use a large dataset
of species interaction networks spanning three contrasted types of ecological
interactions (herbivory, parasitism, and mutualism), to characterize to what
extent species with different specificities can co-occur within the same community.
