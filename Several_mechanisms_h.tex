Several mechanisms have been proposed to explain the co-occurrence of
potentially competing species, including behavior [@abrams_prerequisites_2006],
spatial or temporal heterogeneity [@wilson_coexistence_1994], and
trade-offs associated with species interactions [@mcpeek_trade-offs_1996;
@egas_evolution_2004; @poisot_conceptual_2011]. Ecological factors such as
environmental and spatial heterogeneity and evolutionary processes such as
niche partitioning may permit the coexistence between competing species with
similar and/or different interaction ranges [@bascompte_asymmetric_2006;
@poullain_evolution_2008]. However, most of these results were obtained
in systems of low complexity, and the extent to which specialists and
generalists co-occur in natural communities remains to be evaluated. By
analyzing three bipartite network data sets that cover a range of both
ecological and structural situations, we show how co-occurrence can be
linked with other topological network properties. This calls for a better
integration of network methodology to the analysis of community structure,
so as to evaluate the importance of emerging properties as drivers of the
maintenance of species with different specificities.
