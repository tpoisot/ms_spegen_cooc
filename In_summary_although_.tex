In summary, although the ecological nature of the interaction (mutualistic
or antagonistic) has an impact on network structure, a strategy diversity
that is higher than expected by chance seems to be a conserved property
in bipartite ecological networks. The particular position occupied by a
network along a continuum of, *e.g.* connectance or nestedness, can emerge
because of the life-history traits of species establishing interactions,
and we suggest that increased attention should be given to understanding
how fine- scale mechanisms at the individual or population level drive the
structure of community-level networks. It is nonetheless clear that despite
theoretical predictions, generalists and specialists are often found together
in nature. Understanding this gap between predictions and observations will
be a major challenge, one that should be addressed by understanding the
mechanisms of coexistence and co-occurrence in large multi-species communities.
