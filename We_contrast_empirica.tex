We contrast empirical observations with the predictions of two different
null models, each based on the impact of different aspects of network
structure (connectance, and degree distribution in the whole network). For
each null model, we filled a network through a Bernoulli process, in
which the probability of each pairwise species interaction occurring
($\mathrm{P}\sb{ij}$) is determined in one of the following ways. The first
(connectance based) null model [*e.g.* @fortuna_habitat_2006] assigns the
same probability to each interaction, $\mathrm{P}\sb{ij} = Co$. Compared
to the empirical network on which they are based, simulated networks have
the same connectance, but a potentially different degree distribution. The
second null model [@bascompte_nested_2003] uses information about species
degree to calculate the probability that a particular interaction will
occur. This probability is $\mathrm{P}\sb{ij} = (T\times G\sb{i}+L\times
V\sb{j})/(2\times Z)$, where $G\sb{i}$ and $V\sb{j}$ are, respectively, the
generality of upper trophic level species $i$, and the vulnerability of lower
trophic level species $j$ [@schoener_food_1989]. Simply put, the probability
of the interaction occurring is the mean of the degrees (ranged in 0--1) of the
two species involved. Note that the first null model is nested into the second.
