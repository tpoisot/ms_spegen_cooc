In a previous paper [@poisot_conceptual_2011], we suggested that the
mechanisms shaping the evolution of specificity were similar for both
antagonistic and mutualistic interactions, which implies that relationships
between specificity, the co- occurrence of specialists and generalists, and
other metrics of community structure should be the same across different
types of ecological interaction, despite each category of network having
different structural properties. One central result of the analyses presented
in the present contribution is that empirical data show consistently more
variation in specificities of all species on the upper trophic level
(hereafter called “strategy diversity”) than predicted by two
contrasting null models. This suggests that organisms with very different
levels of specificity do co- occur in most natural systems. Importantly,
we reveal the existence of a continuum from networks of mostly-specialized
to mostly-generalized species, with the potential for specialist/generalist
co-occurrence being greater at intermediate points. Strategy diversity is
contingent upon network properties, including nestedness (a measure of niche
overlap between species with different specificities), modularity (the fact
that species interact within loosely connected clusters), and connectance (the
proportion of realized links), emphasizing the need to adopt a network-oriented
methodology in the study of biotic interactions at the community level.
