@poisot_when_2013 showed that most emergent network properties could be
predicted based on connectance. This included, notably, components of the
degree distribution involved in determining nestedness. The results presented
here suggest that although the relationship between connectance, emergent
metrics (such as nestedness and modularity), and the strategy diversity
is conserved across types of ecological interactions (the strength of this
assertion being tempered by the low sample size for herbivory and mutualism
networks), the difference between types of interactions stemmed mostly from
the networks having different connectances. Specifically, host-parasite
networks were more connected than the other types, but all three types of
interaction had an equal proportion (approximately two-thirds) of networks
with more diversity of strategy than expected by chance. Overall, we report
that networks with a higher nestedness and lower modularity, also had more
strategy diversity than expected under the assumptions of the two null
models. This observation offers promising new research perspectives. If
the main difference between types of interaction is their connectance,
then the different mechanisms involved in different types must be studied
alongside their impacts on network structure. Species specialization is
regulated by differences in life-history traits [@poisot_conceptual_2011],
competition for access to resources [@bascompte_asymmetric_2006;
@bascompte_plant-animal_2007], or phylogenetic conservatism in attack/defense
strategies [@cavender-bares_merging_2009]. Through their impact on species
range of interactions, these factors are likely to be involved in driving
network structure, and connectance in particular. For example, in herbivorous
systems, plants may employ multiple defenses against enemies, including the
release of toxic compounds [@Arimura2005] and/or attraction of a herbivore’s
natural enemies [@Ode2006; @Wei2007; @Van-Nouhuys2003; @Singer2012]. The
simultaneous existence of different levels of defense, relying on chemical
and biotic interactions can promote lower connectance than in other systems,
because there are several steps than the resource can use to prevent the enemy
for interacting with it. It can also result in the faster diversification
of exploitation strategies at the upper level (in the sense that enemies
specialize on a *defense mechanism* rather than on the set of species that
carries it) than in other types of interactions relying on a narrower set
of mechanisms [@Forister2012]. This may result in the maintenance of high
strategy diversity relative to connectance in some antagonistic interactions.
